\chapter{Software installation}
\label{software}


Do not underestimate the effort required to flash your first LED. You
require:

\begin{itemize}
\item
  A computer with \program{git} installed and a useful shell program such as
  \program{bash}. UC has a \href{http://ucmirror.canterbury.ac.nz/}{mirror}
  for a variety of Linux distributions; we recommend Ubuntu or Mint.

\item A cloned copy of the Wacky Racers git repository, see \hyperref[project-cloning]{project cloning}.

\item
  A working ARM toolchain (arm-none-eabi-gcc/g++ version 4.9.3 or newer), see \hyperref[toolchain]{toolchain}.

\item
  An ST-link programmer and 10-wire ribbon cable for programming. You
  can get the adaptor from Scott Lloyd in the SMT lab. You will need
  to make your own cable (For a grey ribbon cable, align the red
  stripe with the small arrow denoting pin 1 on the connector. For a
  rainbow ribbon cable, connect the brown wire to pin 1.). There are
  two variants of the ST-link programmer with \textbf{different
  pinouts} so you may need to customise your programming cable.
\item
  Plenty of gumption.
\end{itemize}


\section{Git}

Your group leader should fork the Wacky Racers project template.  This
creates your own group copy of the project on the eng-git server that
you can modify, add members, etc.

Each group member then clones the group project.


\subsection{Project forking}
\label{project-forking}

The template software is hosted on the eng-git \program{Git} server.
To fork the template:

\begin{enumerate}
\item
  Go to \url{https://eng-git.canterbury.ac.nz/wacky-racers/wacky-racers}.
\item
  Click the `Fork' button. This will create a copy of the main repository
  for the project.
\item
  Click on the `Settings' menu then click the `Expand' button for
  `Sharing and permissions'. Change `Project Visibility' to `Private'.
\item
  Click on the `Members' menu and add group members as Developers.
\end{enumerate}


\subsection{Project cloning}
\label{project-cloning}

Once your project has been forked from the template project, each group
member needs to clone it. This makes a local copy of your project on
your computer. 

If you are using an ECE computer, it is advised that you clone the
project on to a removable USB flash drive. This will make git
operations and compilation 100 times faster than using the networked
file system.

There are two ways to clone the project. If you are impatient and do not
mind having to enter a username and password for every git pull and push
operation use:
%
\begin{minted}[breaklines]{bash}
$ git clone https://eng-git.canterbury.ac.nz/groupleader-userid/wacky-racers.git
\end{minted}

Otherwise, set up \program{ssh-keys} and use:
%
\begin{minted}[breaklines]{bash}
$ git clone git@eng-git.canterbury.ac.nz:groupleader-userid/wacky-racers.git
\end{minted}

You can have several different cloned copies of your project in
different directories. Sometimes if you feel that the world,
and \program{git} in particular, is against you, clone a new copy,
using:
%
\begin{minted}[breaklines]{bash}
$ git clone https://eng-git.canterbury.ac.nz/groupleader-userid/wacky-racers.git wacky-racers-new
\end{minted}


\section{Toolchain}
\label{toolchain}

The toolchain comprises the compiler, linker, debugger, C-libraries,
and OpenOCD.

The toolchain is installed on computers in the ESL and CAE. It should
run under both Linux and Windows.  If there is a problem ask the
technical staff.

The toolchain can be downloaded for Windows, Linux, and macOS from
\url{https://developer.arm.com/tools-and-software/open-source-software/developer-tools/gnu-toolchain/gnu-rm/downloads}.
For Linux and macOS, however, it is better to install the toolchain using your
system's package manager: see the instructions in the following subsections.


\subsection{Toolchain for Linux}

First, if using Ubuntu or Mint, ensure the latest versions are
downloaded:
%
\begin{minted}{bash}
$ sudo apt update && sudo apt upgrade
\end{minted}

Then, install the compiler:
%
\begin{minted}{bash}
$ sudo apt install gcc-arm-none-eabi
\end{minted}

Install the C and C++ libraries:
%
\begin{minted}{bash}    
$ sudo apt install libnewlib-arm-none-eabi libstdc++-arm-none-eabi-newlib
\end{minted}

Install the debugger, GDB:
%
\begin{minted}{bash}    
$ sudo apt install gdb-multiarch
\end{minted}

Install OpenOCD:
%
\begin{minted}{bash}
$ sudo apt install openocd
\end{minted}

\subsection{Toolchain for macOS}

For macOS machines that have \href{https://brew.sh}{homebrew}
installed, you can use the following command:

\begin{minted}{bash}
$ brew install openocd git
$ brew cask install gcc-arm-embedded
\end{minted}


\section{IDE}

I am content with using the command-line and emacs as a text editor
but I guess that you are not.  If you are familiar with geany I
suggest that you use that.  If you want bells and whistles, I suggest
using Visual Studio Code.  While it is written by Microsoft, it is
free and runs on Windows, Linux and macOS.


\subsection{Visual Studio Code}

VS Code is a modern and highly versatile text editor.  With the right
extensions, it can be used to develop code for just about anything!
This project has been configured for VS Code on the ESL computers (but
its relatively easy to modify it to work on a home computer, be it
Windows, Linux, or Mac). For the simplest use, simply go \code{File ->
Open Folder} and point it at the wacky-racers repository.  This will
give you access to the build tools (Ctrl-Shift-B) that have been
setup. These must be run from inside a program (i.e., ledflash1.c).
Finally, opening a program and pressing F5 will launch a debugging
session that will allow the use of breakpoints and variable
inspection.

When building your programs, the \code{BOARD} configuration variable
must be set. In VS Code this is done by choosing your C++
configuration (bottom right of the window) which by default will be
either hat or racer.

As a side note, the compilation and debugging requires the
installation of two VS Code extensions:
%
\begin{enumerate}
\item C/C++ (Microsoft)
\item Native Debug (Web Freak)
\end{enumerate}
%
VS Code should automatically prompt you to install these when you open
  the wacky-racers directory.


\section{Using Windows in the ESL}

On the Windows machines in the ESL, the toolchain is located at
\file{C:\textbackslash{}ence461\textbackslash{}tool-chain}. This needs to be
added to your session's \code{PATH} variable so toolchain commands can be run
without needing to specify their full path. See
\wikiref{ARM-GCC_in_the_DSL}{ARM-GCC in the DSL} on ECEWiki for more
information.

You can use Windows Command Prompt similarly to a Linux Shell. To open, press
\code{Windows Key} + \code{R}, type \code{cmd}, and \code{Enter}. Then run
\file{C:\textbackslash{}ence461\textbackslash{}tool-chain} inside the CMD window
to source the toolchain. Although this guide is written for the Linux shell, the
commands will also work in Command Prompt provided you observe the following
differences:
\begin{itemize}
\item To change between drives (e.g. \file{C:} and \file{D:}), pass the
\code{/d} switch to the \code{cd} command.
\item Instead of \code{export VARIABLE=value}, use \code{set VARIABLE=value}.
Also you cannot set an environment variable when you run a command, so
\code{BOARD=hat make} won't work; run \code{set BOARD=hat} and \code{make} in
separate commands instead.
\item To run GDB, you must invoke the complete command name
\code{arm-none-eabi-gdb}.
\item In path names, \file{/} is replaced by \file{\textbackslash} (backslash).
\end{itemize}

If you are using Windows on the ESL machines and want to use VS Code, you can
run \file{vscode.bat} located in the top level of the \file{wacky-racers}
repository. This will open VS Code with the repository added to the workspace
and the necessary toolchain added to \code{PATH}.
