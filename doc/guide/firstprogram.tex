\chapter{First program}
\label{first-program}

Your first program to test your board should only flash an LED (the
hello world equivalent for embedded systems). The key to testing new
hardware is to have many programs that only do one simple task each.

Before you run your first program, you need to:
%
\begin{enumerate}
\item Install the \hyperref[toolchain]{toolchain}.

\item Clone the Wacky Racers git repository, see
  \hyperref[project-cloning]{project cloning}.

\item Set up the PIO definitions for your board, see
  \hyperref[configuration]{configuration}.

\item Compile your program, see \hyperref[compilation]{compilation}.

\item See if your SAM4S is running, see \hyperref[openocd]{OpenOCD}.

\item Configure the SAM4s to boot from flash memory, see
  \hyperref[booting-from-flash-memory]{booting from flash memory}.

\item Program the SAM4s, see \hyperref[programming]{programming}.
\end{enumerate}


\begin{figure}
\input{figs/openocd.schtex}
\caption{How OpenOCD interacts with the debugger and the SAM4S.}
\label{fig:openocd diagram}
\end{figure}


\section{Connecting with OpenOCD}
\label{openocd}

\program{OpenOCD} is used to program the SAM4S, see \reffig{openocd diagram}.

For this assignment, we use a ST-link programmer to connect to the
SAM4S using serial wire debug (SWD). This connects to your board with
a 10-wire ribbon cable and an IDC connector.

\begin{enumerate}
\item
  Before you start, disconnect the battery and other cables from your
  PCB.
\item
  Connect a 10-wire ribbon cable from the ST-link programmer to the
  programming header on your PCB. This will provide 3.3\,V to your
  board so your green power LED should light.
\item
  Open a \textbf{new terminal window, e.g., bash} and
  start \program{OpenOCD}.
\end{enumerate}

\begin{minted}{bash}
$ cd wacky-racers
$ openocd -f src/mat91lib/sam4s/scripts/sam4s_stlink.cfg
\end{minted}

All going well, the last line output from \program{OpenOCD} should be:

\begin{verbatim}
Info : sam4.cpu: hardware has 6 breakpoints, 4 watchpoints
\end{verbatim}

Congrats if you get this! It means you have correctly soldered your
SAM4S. If not, do not despair and do not remove your SAM4S. Instead,
see \protect\hyperref[troubleshooting]{troubleshooting}.


\section{LED flash program}
\label{led-flash-program}

For your first program, use
\wfile{test-apps/ledflash1/ledflash1.c}. The macros \code{LED1_PIO}
and \code{LED2_PIO} need to be defined in \file{target.h} (see
\protect\hyperref[configuration]{configuration}).

\inputminted{C}{../../src/test-apps/ledflash1/ledflash1.c}

The LED should flash at 2\,Hz.  If it does not, try the
\hyperref[pwm-test]{PWM test} program and check the frequency with an
oscilloscope.


\section{Configuration}
\label{configuration}

Each board has different PIO definitions and requires its own
configuration information. The \wfile{boards} directory contains a
configuration for the hat and one for the racer board.  \textbf{You
must edit these to customise your board.}  Each configuration
directory contains three files:

\begin{itemize}
\item
  \file{board.mk} is a makefile fragment that specifies the MCU model,
  optimisation level, etc.
\item
  \file{target.h} is a C header file that defines the PIO pins and
  clock speeds.
\item
  \file{config.h} is a C header file that wraps target.h. Its purpose
  is for porting to different compilers.
\end{itemize}

\textbf{You will need to edit the target.h file for your board} and set
the definitions appropriate for your hardware. Here's an excerpt from
\file{target.h} for a hat board:

\begin{minted}{C}
/* USB  */
#define USB_VBUS_PIO PA5_PIO

/* ADC  */
#define ADC_BATTERY PB3_PIO
#define ADC_JOYSTICK_X PB2_PIO
#define ADC_JOYSTICK_Y PB1_PIO

/* IMU  */
#define IMU_INT_PIO PA0_PIO

/* LEDs  */
#define LED1_PIO PA20_PIO
#define LED2_PIO PA23_PIO
\end{minted}


\section{Compilation}
\label{compilation}

Due to the many files required, compilation is performed using
makefiles.

The demo test programs are generic and you need to specify which board
you are compiling them for. The board configuration file can be chosen
dynamically by defining the environment variable \code{BOARD}. For
example:
%
\begin{minted}{bash}
$ cd src/test-apps/ledflash1
$ BOARD=racer make
\end{minted}

If all goes well, you should see at the end:
%
\begin{verbatim}
   text    data     bss     dec     hex filename
  11348	   2416	    176	  13940	   3674	ledflash1.bin
\end{verbatim}

To avoid having to specify the environment variable \code{BOARD}, you
can define it for the rest of your session using:
%
\begin{minted}{bash}
$ export BOARD=racer
\end{minted}
%
and then just use:
%
\begin{minted}{bash}
$ make
\end{minted}

\section{Booting from flash memory}
\label{booting-from-flash-memory}

By default the SAM4S runs a bootloader program stored in ROM. The SAM4S
needs to be configured to run your application from flash memory.

If \program{OpenOCD} is running you can do this with:

\begin{minted}{bash}
$ make bootflash
\end{minted}

Unless you force a complete erasure of the SAM4S flash memory by
connecting the \pin{ERASE} pin to 3.3\,V, you will not need to repeat
this command.

\section{Programming}
\label{programming}

If \program{OpenOCD} is running you can store your program in the flash
memory of the SAM4S using:

\begin{minted}{bash}
$ make program
\end{minted}

When this finishes, one of your LEDs should flash. If so, congrats! If
not, see \protect\hyperref[troubleshooting]{troubleshooting}.

To reset your SAM4S, you can use:
%
\begin{minted}{bash}
$ make reset
\end{minted}


\section{Makefiles}

Compilation is performed using makefiles, since each application
requires many files.  Rather than having large makefiles, a heirarchy
of makefile fragments is employed.  Dependencies are automatically
generated (so not all the files have to be recompiled each time).

A board description can be selected with the \code{BOARD} environment
variable.  This can be defined for each command, for example,
%
\begin{minted}{bash}
$ BOARD=racer make program
\end{minted}


Alternatively, this can be defined for a session:
%
\begin{minted}{bash}
$ export BOARD=racer
$ make program
\end{minted}


Each of the makefiles has the following phony targets:
%
\begin{description}
\item[all]  --- compile the application
\item[program] --- compile the application and download to the MCU (you need Openocd running)
\item[debug] --- start the debugger GDB (you need Openocd running)
\item[reset] --- reset the MCU (you need Openocd running)
\item[clean] --- delete the executable, object files, and dependency files.
\end{description}
